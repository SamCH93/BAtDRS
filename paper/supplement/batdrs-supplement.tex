\documentclass[a4paper, 11pt]{article}
\usepackage[dvipsnames]{xcolor}
%% packages
%% ----------------------------------------------------------------------------
\usepackage[T1]{fontenc}
\usepackage[utf8]{inputenc}
\usepackage[english]{babel}
\usepackage{amsmath, amssymb}
\usepackage{doi} % automatic doi-links
\usepackage[round]{natbib} % bibliography
\usepackage{multirow} % multi rows and columns for tables
\usepackage{longtable} % tables that span several pages
\usepackage{booktabs} % nicer tables
\usepackage[title]{appendix} % better appendices
\usepackage{nameref} % reference appendices with names

% fonts

% \usepackage{lmodern} % math and serif font
% \usepackage{mathpazo} % math and serif font
\usepackage{times} % math and serif font
\usepackage[semibold]{sourcesanspro} % sans serif font
\usepackage{sectsty} % use different fonts for different sections
\allsectionsfont{\sffamily} % for sections use sans serif
\usepackage[onehalfspacing]{setspace} % more space
\usepackage[labelfont=bf,font=small]{caption} % smaller captions


% tikz options
\usepackage{tikz}
\usetikzlibrary{positioning,shapes,fit}
% define color for tikz figure
\definecolor{lightlightgray}{RGB}{211,211,211}
 % set default tikz font to sans serif
\tikzset{every picture/.style={/utils/exec={\sffamily}}}

%% margins
%% ----------------------------------------------------------------------------
\usepackage{geometry}
\geometry{
  a4paper,
  total={170mm,257mm},
  left=25mm,
  right=25mm,
  top=20mm,
  bottom=20mm,
}

% %% title, authors, affiliations, mail
% %% ----------------------------------------------------------------------------
\newcommand\longtitle{Bayesian approaches to designing replication studies}
\newcommand\shorttitle{\longtitle} % if longtitle too long, change here
\newcommand\longauthors{Samuel Pawel\textsuperscript{$\star$}, Guido
  Consonni\textsuperscript{$\dagger$}, Leonhard
  Held\textsuperscript{$\star$}}
\newcommand\shortauthors{S. Pawel, G. Consonni,
  L. Held} % if longauthors too long, change here
\newcommand\affiliation{
  $\star$ Department of Biostatistics, University of Zurich \\
  $\dagger$ Dipartimento di Scienze Statistiche, Universit\`{a} Cattolica del Sacro
  Cuore } \newcommand\mail{samuel.pawel@uzh.ch} \title{ \vspace{-2em}
  \textbf{\longtitle} } \author{
  \textbf{\longauthors} \\
  \affiliation \\
  E-mail: \href{mailto:\mail}{\mail} }
  \date{\today} % don't forget to hard-code date when submitting to arXiv!
% \title{Bayesian approaches to designing replication studies}
% \shorttitle{Bayesian approaches to designing replication studies}
% \author{Samuel Pawel}
% \date{\today} % don't forget to hard-code date when submitting to arXiv!

%% hyperref options
%% ----------------------------------------------------------------------------
\usepackage{hyperref}
\hypersetup{
  bookmarksopen=true,
  breaklinks=true,
  pdftitle={\shorttitle},
  pdfauthor={\shortauthors},
  pdfsubject={},
  pdfkeywords={},
  colorlinks=true,
  linkcolor=blue,
  anchorcolor=black,
  citecolor=blue,
  urlcolor=black,
}

%% Headers and footers
%% ----------------------------------------------------------------------------
\usepackage{fancyhdr}
\pagestyle{fancy}
\lhead{\shorttitle}
\rhead{\shortauthors}

%% Definitions
%% ----------------------------------------------------------------------------
%% Definitions
%% -----------------------------------------------------------------------------
%% Shortforms
\newcommand{\eg}{e.\,g.,} % e.g.
\newcommand{\ie}{i.\,e.,} % i.e.
%% Distributions
\DeclareMathOperator{\Ber}{B} % Bernoulli
\DeclareMathOperator{\Bin}{Bin} % Binomial Distribution
\DeclareMathOperator{\Cauchy}{C} % Cauchy Distribution
\DeclareMathOperator{\Po}{Po} % Poisson
\DeclareMathOperator{\Exp}{Exp} % Exponential
\DeclareMathOperator{\Nor}{N} % Normal
\DeclareMathOperator{\stud}{t} % Student
\DeclareMathOperator{\Ga}{G} % Gamma
\DeclareMathOperator{\Be}{Be} % Beta
%% Operators and special functions
\DeclareMathOperator{\Var}{Var} % Variance
\DeclareMathOperator{\E}{\mathsf{E}} % Expectation
\DeclareMathOperator{\Cov}{Cov} % Covariance
\DeclareMathOperator{\Corr}{Corr} % Correlation
\DeclareMathOperator{\se}{se} % Standard error
\DeclareMathOperator{\sign}{sign} % Sign
\DeclareMathOperator{\logit}{logit} % Logit
\DeclareMathOperator{\Mod}{Mod} % Modus
\DeclareMathOperator{\Med}{Med} % Median
\DeclareMathOperator{\diag}{diag} % Diagonalmatrix
\DeclareMathOperator{\trace}{tr} % Trace
\renewcommand{\P}{\operatorname{\mathsf{Pr}}} % Probability
\newcommand{\p}{f} % Density function
\newcommand{\B}{\operatorname{{B}}} % Beta function
\newcommand{\Lik}{L} % Likelihood function
\DeclareMathOperator{\arctanh}{arctanh} % Arcus tangens hyperbolicus
\DeclareMathOperator*{\argmax}{arg\,max} % Argmax
\DeclareMathOperator*{\argmin}{arg\,min} % Argmin
%% Other math things
\newcommand{\C}{\mathbb{C}} % Complex
\newcommand{\R}{\mathbb{R}} % Reals
\newcommand{\Q}{\mathbb{Q}} % Rationals
\newcommand{\Z}{\mathbb{Z}} % Integers
\newcommand{\N}{\mathbb{N}} % Naturals
\newcommand{\given}{\,\vert\,} % Given
\newcommand{\abs}[1]{\left\lvert#1\right\rvert} % Absolutvalue
\newcommand{\norm}[1]{\left\lVert#1\right\rVert} % Norm
\newcommand{\ceil}[1]{\left\lceil#1\right\rceil} % Ceiling
\newcommand{\floor}[1]{\left\lfloor#1\right\rfloor} % Floor
\newcommand{\sprod}[1]{\left\langle#1\right\rangle} % Scalarproduct
\newcommand{\Ind}[2]{\mathsf{I}_{#2}(#1)} % Indicatorfunction
\newcommand{\IdMat}{\boldsymbol{\mathrm{I}}} % Identity matrix
\DeclareMathOperator{\BF}{BF} % Bayes factor
\newcommand\BFsub[2]{\BF_{\scriptscriptstyle{#1:#2}}} % BF with subscript
\newcommand{\that}{\hat{\theta}} % Effect estimate
\newcommand{\lw}[1]{W_{\scriptscriptstyle{#1}}} % Lambert W function
\newcommand{\BFs}{\text{BF}_{\scriptscriptstyle\text{S}}} % Sceptical BF
\newcommand{\BFr}{\text{BF}_{\scriptscriptstyle \text{R}}} % Replication BF
\newcommand{\ps}{p_{\scriptscriptstyle\text{S}}} % Sceptical p-value
\newcommand{\zas}{z_{\alpha_{\text{S}}}} % quantile of sceptical p-value level
\newcommand{\gy}{s_{\scriptscriptstyle \gamma}}
\newcommand{\HS}{H_{\text{S}}}
\newcommand{\HA}{H_{\text{A}}}
\newcommand{\mutheta}{\mu_{\scriptscriptstyle \theta}}
\newcommand{\sigmatheta}{\sigma_{\scriptscriptstyle \theta}}
\newcommand{\muthatr}{\mu_{\scriptscriptstyle \hat{\theta}_{r}}}
\newcommand{\sigmathatr}{\sigma_{\scriptscriptstyle \hat{\theta}_{r}}}
\newcommand{\muzr}{\mu_{\scriptscriptstyle z_{r}}}
\newcommand{\sigmazr}{\sigma_{\scriptscriptstyle z_{r}}}
\newcommand{\zalpha}{z_{\scriptscriptstyle \alpha}}
\newcommand{\zalphatwo}{z_{\scriptscriptstyle \alpha/2}}
\newcommand{\zbeta}{z_{\scriptscriptstyle \beta}}
\newcommand{\bthat}{\hat{\boldsymbol{\theta}}} % bold effect estimate
\newcommand{\bsigma}{\boldsymbol{\sigma}} % bold sigma
\newcommand{\btheta}{\boldsymbol{\theta}} % bold theta
\newcommand{\bone}{\mathbf{1}} % bold 1
\newcommand{\STTR}{S_{\scriptscriptstyle \text{2TR}}}
\newcommand{\SMA}{S_{\scriptscriptstyle \text{MA}}}
\newcommand{\SEqu}{S_{\scriptscriptstyle \text{E}}}
\newcommand{\SQ}{S_{\scriptscriptstyle \text{Q}}}
\newcommand{\SPs}{S_{\scriptscriptstyle \ps}}
\newcommand{\SBFr}{S_{\scriptscriptstyle \BFr}}
\newcommand{\SBFs}{S_{\scriptscriptstyle \BFs}}



\begin{document}
\maketitle

In this document we provide additional information on methods for analyzing
replication data. For each method we also derive the \emph{success region} in
terms of the effect estimate of the replication study $\that_{r}$, which is
required for sample size determination as illustrated in the main manuscript.
For the two-trials rule and the replication Bayes factor methods we additionally
provide derivations on how these methods can be generalized to the multisite
replication setting.

% \begin{table}[!htb]
%   \centering
%   \caption{Notation for data from original and replication study.}
%   \begin{tabular}{lcc}
%     \toprule
%     & Original study & Replication study \\
%     Effect estimate & $\that_o$ & $\that_{r}$ \\
%     Standard error & $\sigma_{o}$ & $\sigma_{r}$ \\
%     $z$-value & $z_{o} = \that_{o}/\sigma_{o}$ & $z_{r} = \that_{r}/\sigma_{r}$ \\
%     \bottomrule
%   \end{tabular}
% \end{table}

\section{The two-trials rule}
The two-trials rule is the most common analysis approach for replication
studies. Replication success is declared if both original and replication study
achieve statistical significance at some level $\alpha$ (and both estimates go
in the same direction which can be taken into account by using one-sided
$p$-values). We will study the two-trial under normality using the data model
$\that_{i} \given \theta \sim \Nor(\theta, \sigma^{2}_{i})$ with $\that_{i}$ the
estimate of the unknown effect size $\theta$ from study $i$ and $\sigma_{i}$ is
the corresponding standard error (assumed to be know). The $p$-values for
testing $H_{0} \colon \theta = 0$ versus $H_{1} \colon \theta > 0$ are then
$p_{i} = 1 - \Phi(\that_{i}/\sigma_{i})$ whereas for the alternative
$H_{1} \colon \theta < 0$ they are $p_{i} = \Phi(\that_{i}/\sigma_{i})$. Suppose
the original effect estimate was statistically significant at level $\alpha$,
\ie{} $p_{o} \leq \alpha$. Replication success at level $\alpha$ is then
established if the replication effect estimate $\that_{r}$ is also statistically
significant at level $\alpha$, \ie{} $p_{r} \leq \alpha$. By applying some
algebraic manipulations to the success condition, one can show that this implies
that replication success is achieved if the replication effect estimate
$\that_{r}$ is contained in the success region
\begin{align*}
  S_{\text{2TR}} =
  \begin{cases}
    \left[\zalpha \, \sigma_{r}, \infty \right) & \text{for} ~ \that_{o} > 0 \\
    \left[-\infty, -\zalpha \, \sigma_{r} \right) & \text{for} ~ \that_{o} < 0.
  \end{cases}
\end{align*}

\subsection{The multisite two-trials rule}
If multiple replication studies are conducted for one original study (a
\emph{multisite} replication), the two-trials rule is typically modified by
meta-analyzing the effect estimates from all replications and then using the
combined estimate as usual in the two-trials rule \citep[see \eg{} the ``Many
labs'' projects from][]{Klein2014, Klein2018}. Suppose $m$ replication studies
are conducted and produce $m$ effect estimates $\that_{r1}, \dots, \that_{m}$
with standard errors $\sigma_{r1}, \dots, \sigma_{rm}$. Subsequently, a weighted
average
$\hat{\theta}_{r*} = \{\sum_{i = 1}^{m} \hat{\theta}_{ri}/(\sigma^{2}_{ri} + \tau^{2}_{r})\}\,\sigma_{r*}^{2}$
with standard error
$\sigma_{r*} = 1/\surd\{\sum_{i}^{m}1/(\sigma^{2}_{ri} + \tau^{2}_{r})\}$ can be
computed. If the between-replication heterogeneity variance $\tau^{2}_{r}$ is
set to zero this corresponds to the fixed effects estimate of $\theta$, while
estimating $\tau^{2}_{r}$ from the data corresponds to the random effects
estimate. Replication success at level $\alpha$ is then established if the
replication $p$-value is smaller than $\alpha$, \ie{}
$p_{r*} = 1 - \Phi(\that_{r*}/\sigma_{r*}) \leq \alpha$. With some algebra one
can show that this implies a success region for the weighted average replication
effect estimate $\that_{r*}$ given by
\begin{align*}
  S_{\text{2TR}} =
  \begin{cases}
    \left[\zalpha \, \sigma_{r*}, \infty \right) & \text{for} ~ \that_{o} > 0 \\
    \left[-\infty, -\zalpha \, \sigma_{r*} \right) & \text{for} ~ \that_{o} < 0.
  \end{cases}
\end{align*}

\section{Fixed effects meta-analysis}
Assume again the data model
$\that_{i} \given \theta \sim \Nor(\theta, \sigma^{2}_{i})$ where $\that_{i}$ is
an estimate of the effect size $\theta$ from study $i \in \{o, r\}$ and
$\sigma_{i}$ is the corresponding standard error (assumed to be know). In the
fixed effects meta-analysis approach replicability is assessed in terms of the
pooled effect estimate $\that_{m}$ and standard error $\sigma_{m}$ which are
\begin{align*}
  &\that_{m} =
    \left(\that_{o}/\sigma_{o}^{2} + \that_{r}/\sigma^{2}_{r}\right)\sigma^{2}_{m}&
&\text{and}&                                                                                       &\sigma_{m} = \left(1/\sigma^{2}_{o} + 1/\sigma^{2}_{r}\right)^{-1/2},&
\end{align*}
which are also equivalent to the mean and standard deviation of a posterior
distribution for the effect size $\theta$ based on the data from original and
replication study and an initial flat prior for $\theta$. Fixed effects
meta-analyis is typically used because estimating a heterogeneity variance from
two studies is highly unstable. Replication success at level $\alpha$ is
established if the one-sided meta-analytic $p$-value (in the direction of the
original effect estimate $\that$) is significant at level $\alpha$, \ie{}
$p_{m} = 1 - \Phi(\that_{m}/\sigma_{m}) \leq \alpha$ for $\that_{o} > 0$ and
$p_{m} = \Phi(\that_{m}/\sigma_{m}) \leq \alpha$ for $\that_{o} < 0$.
With some algebraic manipulations one can show that this criterion implies a
success region $S_{\text{MA}}$ for the replication effect estimate $\that_{r}$
given by
\begin{align*}
  \SMA
  =
  \begin{cases}
    [\sigma_{r} \zalpha\sqrt{1 + \sigma^{2}_{r}/\sigma^{2}_{o}} -
      (\that_{o} \sigma^{2}_{r})/\sigma^{2}_{o},   \infty )
    & \text{for} ~ \that_{o} > 0 \\
    (-\infty, -\sigma_{r} \zalpha\sqrt{1 + \sigma^{2}_{r}/\sigma^{2}_{o}} -
      (\that_{o} \sigma^{2}_{r})/\sigma^{2}_{o}]
    & \text{for} ~ \that_{o} < 0.
\end{cases}
\end{align*}


\section{Effect size equivalence}
The effect size equivalence approach \citep{Anderson2016} defines replication
success via comptability of the effect estimates from both studies. Under
normality we may assume the data model
$\that_{i} \given \theta_{i} \sim \Nor(\theta_{i}, \sigma^{2}_{i})$ for study
$i \in \{o, r\}$, and we are interested in the true effect size difference
$\delta = \theta_{r} - \theta_{o}$. A $(1 - \alpha)$ confidence interval for
$\delta$ is then given by
\begin{align*}
  C_{\alpha} = \left[\that_{r} - \that_{o} - \zalphatwo \sqrt{\sigma^{2}_{r} + \sigma^{2}_{r}},
  \that_{r} - \that_{o} + \zalphatwo \sqrt{\sigma^{2}_{r} + \sigma^{2}_{r}}\right]
\end{align*}
Effect size equivalence is established if the confidence interval is fully
included in an equivalence region $C_{\alpha} \subseteq [-\Delta, \Delta]$ with
$\Delta > 0$ a pre-specified margin. Applying some algebraic manipulations to
the success condtions one can show that the equivalence test replication success
criterion implies a success region $\SEqu$ for the replication estimate
$\that_{r}$ given by
\begin{align*}
  \SEqu
  = \left[\that_{o} - \Delta + \zalphatwo \sqrt{\sigma^{2}_{o} +
  \sigma^{2}_{r}}, \that_{o} + \Delta - \zalphatwo
  \sqrt{\sigma^{2}_{o} + \sigma^{2}_{r}}\right].
\end{align*}



% \section{The \textit{Q}-test}

\section{The replication Bayes factor}
The replication Bayes factor approach uses the replication data $x_{r}$ to
quantify the evidence for the null hypothesis $H_{0}\colon \theta = 0$ relative
to the alternative hypothesis $H_{1} \colon \theta \sim f(\theta \given x_{o})$,
which postulates that the effect size $\theta$ is distributed according to its
posterior distribution based on the original data $x_{o}$. Assume again a normal
model $\that_{i} \given \theta \sim \Nor(\theta, \sigma^{2}_{i})$ with
$\that_{i}$ an estimate of the effect size $\theta$ from study $i \in \{o, r\}$
and $\sigma_{i}$ the corresponding standard error (assumed to be know), and that
we use the alternative $H_{1} \colon \Nor(\that_{o}, \sigma^{2}_{o})$ which
arises from updating an inital flat prior for $\theta$ the original data
$x_{o} = \{\that_{o}, \sigma_{o}\}$. The replication Bayes factor is then
\begin{align}
  \label{eq:bfr}
  \BFr &= \frac{f(\that_{r} \given H_{0})}{f(\that_{r} \given H_{1})}
       = \sqrt{1 + \sigma^{2}_{o}/\sigma^{2}_{r}} \, \exp\left[
         -\frac{1}{2}\left\{ \frac{\that^{2}_{r}}{\sigma^{2}_{r}} -
         \frac{(\that_{r} - \that_{o})^{2}}{\sigma^{2}_{o} + \sigma^{2}_{r}}
         \right\}\right].
\end{align}
Replication success at level $\gamma \in (0, 1)$ is achieved if
$\BFr \leq \gamma$. By applying some algebra to $\BFr \leq \gamma$, one can show
that it is equivalent to the replication effect estimate $\that_{r}$ falling in
the success region
\begin{align*}
  \SBFr
  = \left(-\infty, -\sqrt{A} - (\that_{o}\sigma^{2}_{r})/\sigma^{2}_{o}\right] \bigcup
   \left[\sqrt{A} - (\that_{o}\sigma^{2}_{r})/\sigma^{2}_{o}, \infty\right)
\end{align*}
where
$A = \sigma^{2}_{r}(1 + \sigma^{2}_{r}/\sigma^{2}_{o}) \{\that_{o}^{2}/\sigma^{2}_{o} - 2 \log \gamma + \log(1 + \sigma^{2}_{o}/\sigma^{2}_{r})\}$.

\subsection{The multisite replication Bayes factor}
The generalization of the replication Bayes factor to the multisite setting is
straightforward. The data are represented by vector of replication effect
estimates $\bthat_{r} = (\that_{r1}, \dots, \that_{rm})^{\top}$ with
corresponding standard error vector
$\bsigma_{r} = (\sigma_{r1}, \dots, \sigma_{rm})^{\top}$, and we assume the data
model
$\bthat_{r} \given \theta \sim \Nor_{m}\{\theta \, \bone_{m}, \diag(\bsigma^{2} + \tau^{2}_{r} \, \bone_{m}\}$
where $\bone_{m}$ is a vector of $m$ ones and $\tau^{2}_{r}$ is a heterogeneity
variance for the replication effect sizes (not to be confused with the
heterogeneity variance $\tau^{2}$ used in the design prior).

As in the singlsite case, the replication Bayes factor quantifies the evidence
that the data provide for the null hypothesis $H_{0}\colon \theta = 0$ relative
to the alternative hypothesis
$H_{1} \colon \theta \sim \Nor(\that_{o}, \sigma^{2}_{o})$. The marginal density
of the replication data under the null hypothesis is simply
$\bthat_{r} \given H_{0} \sim \Nor_{m}\{0 \, \bone_{m}, \diag(\bsigma^{2} + \tau^{2}_{r} \, \bone_{m}\}$,
whereas the marginal likelihood under the alternative $H_{1}$ is obtained from
integrating the likelihood
% $\bthat_{r} \given \theta \sim \Nor_{m}\{\theta \, \bone_{m}, \diag(\bsigma^{2} + \tau^{2}_{r} \, \bone_{m}\}$
with respect to the prior distribution of $\theta$ under the alternative
$H_{1}$. % $H_{1} \colon \theta \sim \Nor(\that_{o}, \sigma_{o})$.
Let $\Nor(x;m,v)$ denote the normal density function mean $m$ and variance $v$
evaluated at $x$. Define also
$\hat{\theta}_{r*} = \left\{\sum_{i=1}^{n}\hat{\theta}_{ri}/(\sigma^{2}_{ri} + \tau^{2}_{r})\right\} \sigma^{2}_{r*}$
and
$\sigma^{2}_{r*} = 1/\left\{\sum_{i=1}^{n}1/(\sigma^{2}_{ri} + \tau^{2}_{r})\right\}$,
\ie{} the weighted average of the replication effect estimates based on the
heterogeneity $\tau^{2}_{r}$ and its variance. The marginal density is then

\begin{align*}
        f(\hat{\theta}_{r} \given H_{1})
        &= \int f(\hat{\theta}_{r} \given \theta) f(\theta \given H_{1})
                        \, \text{d}\theta \\
              &= \int \frac{\exp\left[-\frac{1}{2} \left\{\sum_{i=1}^{n} \frac{(\hat{\theta}_{ri} - \theta)^{2}}{\sigma^{2}_{ri} + \tau^{2}_{r}} +
                  \frac{(\theta - \that_{o})^{2}}{\sigma^{2}_{o}}\right\} \right]}{
                                    \left\{2\pi \sigma^{2}_{o} \prod_{i = 1}^{n} 2\pi \left(\sigma^{2}_{ri} + \tau^{2}_{r}\right)\right\}^{1/2}}
    \, \text{d}\theta \\
  &= \int \frac{
    \exp\left[-\frac{1}{2} \left\{\sum_{i=1}^{n} \frac{(\hat{\theta}_{ri} - \hat{\theta}_{r*})^{2}}{\sigma^{2}_{ri} + \tau^{2}_{r}} +  \frac{(\hat{\theta}_{r*} - \theta)^{2}}{\sigma^{2}_{r*}} +
    \frac{(\theta - \that_{o})^{2}}{\sigma^{2}_{o}}\right\} \right]}{
    \left\{2\pi \sigma^{2}_{o} \prod_{i = 1}^{n} 2\pi \left(\sigma^{2}_{ri} + \tau^{2}_{r}\right)\right\}^{1/2}}
    \, \text{d}\theta \\
  &= \frac{\exp\left[-\frac{1}{2} \left\{\sum_{i=1}^{n} \frac{(\hat{\theta}_{ri} - \hat{\theta}_{r*})^{2}}{\sigma^{2}_{ri} + \tau^{2}_{r}}\right\} \right]}{\left\{2\pi \sigma^{2}_{o} \prod_{i = 1}^{n} 2\pi \left(\sigma^{2}_{ri} + \tau^{2}_{r}\right)\right\}^{1/2}}
    \underbrace{\int \exp\left[-\frac{1}{2} \left\{\frac{(\hat{\theta}_{r*} - \theta)^{2}}{\sigma^{2}_{r*}} +
    \frac{(\theta - \that_{o})^{2}}{\sigma^{2}_{o}}\right\} \right] \text{d}\theta}_{= \Nor(\hat{\theta}_{r*}; m, \sigma^{2}_{o} + \sigma^{2}_{r*}) 2\pi \sigma_{o} \sigma_{r*}} \\
  &= \left\{(1 + \sigma^{2}_{o}/\sigma^{2}_{r*}) \prod_{i = 1}^{n} 2\pi \left(\sigma^{2}_{ri} + \tau^{2}_{r}\right)\right\}^{-1/2} \exp\left[-\frac{1}{2}\left\{
    \sum_{i=1}^{n} \frac{(\hat{\theta}_{ri} - \hat{\theta}_{r*})^{2}}{\sigma^{2}_{ri} + \tau^{2}_{r}} + \frac{(\hat{\theta}_{r*} - \that_{o})^{2}}{\sigma^{2}_{r*} + \sigma^{2}_{o}}\right\}\right].
\end{align*}
Dividing the marginal density of $\bthat_{r}$ under $H_{0}$ by the marginal
density of $\bthat_{r}$ under $H_{1}$ leads to cancelation of several terms, and
produces the replication Bayes factor
\begin{align*}
  \BF_{01}(\hat{\theta}_{r})
  &= \frac{f(\hat{\theta}_{r} \given H_{0})}{f(\hat{\theta}_{r} \given H_{1})}
    = \sqrt{1 + \sigma^{2}_{o}/\sigma^{2}_{r*}}  \exp\left[-\frac{1}{2}\left\{
    \frac{\hat{\theta}_{r*}^{2}}{\sigma^{2}_{r*}} -
    \frac{(\hat{\theta}_{r*} - \hat{\theta}_{o})^{2}}{\sigma^{2}_{r*} + \sigma^{2}_{o}}\right\}\right].
\end{align*}
The multisite replication Bayes factor is therefore equivalent to the singlesite
replication Bayes factor from~\eqref{eq:bfr} but using the weighted average
$\that_{r*}$ and its standard error $\sigma_{r*}$ as the replication effect
estimate $\that_{r}$ and standard error $\sigma_{r}$.

\section{The sceptical \textit{p}-value}
\citet{Held2020} proposed a reverse-Bayes approach for assessing replicability.
One assumes again the data model
$\that_{i} \given \theta \sim \Nor(\theta, \sigma^{2}_{i})$ with
$i \in \{o, r\}$, along with a zero-mean ``sceptical'' prior
$\theta \sim \Nor(0, \sigma^{2}_{s})$ for the effect size. In a first step, a
level $\alpha \geq p_{o} = 1 - \Phi(|\that_{o}|/\sigma_{o})$ is fixed and the
``sufficiently sceptical'' prior variance $\sigma^{2}_{s}$ is computed
\begin{align*}
  \sigma^{2}_{s} = \frac{\sigma^{2}_{o}}{(z_{o}^{2}/\zalpha^{2}) - 1}
\end{align*}
where $z_{o} = \that_{o}/\sigma_{o}$. The sufficiently sceptical prior variance
$\sigma^{2}_{s}$ has the property that it renders the resulting posterior of
$\theta$ no longer ``credible'' at level $\alpha$, that is, the posterior tail
probability is fixed to
$\Pr(\theta \geq 0 \given \that_{o}, \sigma_{o}, \sigma_{s}) = 1 - \alpha$ for
positive estimates and
$\Pr(\theta \leq 0 \given \that_{o}, \sigma_{o}, \sigma_{s}) = 1 - \alpha$ for
negative estimates. In a second step, the conflict between the sceptical prior
and the obsereved replication data is quantified, larger conflict indicating a
higher degree of replication success. For doing so, a prior predictive tail
probability
\begin{align*}
  p_{\text{Box}} =
  \begin{cases}
    1 - \Phi\left\{\that_{r}/(\sigma^{2}_{r} + \sigma^{2}_{s})\right\}
    & \text{if} ~ \that_{o} > 0 \\
   \Phi\left\{\that_{r}/(\sigma^{2}_{r} + \sigma^{2}_{s})\right\}
    & \text{if} ~ \that_{o} < 0 \\
    \end{cases}
\end{align*}
is computed and replication success at level $\alpha$ is declared if
$p_{\text{Box}} \leq \alpha$. The smallest level $\alpha$ at which replication
success is achieved is called the \emph{the sceptical $p$-value} $\ps$ and
replication success at level $\alpha$ is equivalent with $\ps \leq \alpha$
\citep[see][for more details on $\ps$]{Held2020, Held2021}. By applying some
algebraic maniputations to the condition $p_{\text{Box}} \leq \alpha$, one can
show that it is equivalent to the replication effect estimate $\that_{r}$
falling in the success region
\begin{align*}
  \SPs =
  \begin{cases}
    [\zalpha \surd\{\sigma^{2}_{r} +
  \frac{\sigma^{2}_{o}}{(z_{o}^{2}/\zalpha^{2}) - 1}\}, \infty)
    & \text{if} ~ \that_{o} > 0 \\
   (-\infty, -\zalpha \surd\{\sigma^{2}_{r} +
  \frac{\sigma^{2}_{o}}{(z_{o}^{2}/\zalpha^{2}) - 1}\}]
    & \text{if} ~ \that_{o} < 0. \\
    \end{cases}
\end{align*}

\section{The sceptical Bayes factor}
\citet{Pawel2022b} modified the reverse-Bayes assessment of replication success
from \citet{Held2020} to use Bayes factors \citep{Jeffreys1961, Kass1995}
instead of tail probabilities as measures of evidence and prior data conflict.
The procedure assumes again the data model
$\that_{i} \given \theta \sim \Nor(\theta, \sigma^{2}_{i})$ for study
$i \in \{o, r\}$. In the first step the original data are used to contrast the
evidence for the point null hypothesis $H_{0} \colon \theta = 0$ relative to the
``sceptical'' alternative $H_{S} \colon \theta \sim \Nor(0, \sigma^{2}_{s})$
with the Bayes factor
\begin{align*}
  \BF_{0S}
  = \frac{f(\that_{o} \given H_{0})}{f(\that_{o} \given H_{S})}
  = \sqrt{1 + \sigma^{2}_{s}/\sigma^{2}_{o}} \, \exp\left\{-
  \, \frac{z_{o}^{2}}{2(1 + \sigma^{2}_{o}/\sigma^{2}_{s})}\right\}.
\end{align*}
where $z_{o} = \that/\sigma^{2}_{o}$. One then determines the sufficiently
sceptical prior variance $\sigma^{2}_{s}$ so that the Bayes factor is fixed to a
level $\gamma \in (0, 1)$ meaning that there is no longer evidence against the
null hypothesis at level $\gamma$. The sufficiently sceptical prior variance can
be computed by
\begin{align}
  \label{eq:ssv}
  \sigma^{2}_{s} &=
  \begin{cases}
    -\dfrac{\that_{o}^2}{q} - \sigma^{2}_{o} & ~~ \text{if} ~ -\dfrac{\that_o^2}{q} \geq \sigma^{2}_{o} \\
    \text{undefined} & ~~ \text{else}
  \end{cases} \\
  \text{where} ~ q &= \lw{-1} \left\{-\frac{z_o^2}{\gamma^2} \,
  \exp\left(-z_o^2\right)\right\}
\end{align}
with $\lw{-1}(\cdot)$ the branch of the
Lambert W function with $\text{W}(y) \leq -1$ for $y \in [-1/e, 0)$.

In a second step the conflict between the sceptical prior and the replication
data is quantified. To do so, the sceptic is contrasted to the ``advocacy''
alternative $H_{A} \colon \theta \sim \Nor(\that_{o}, \sigma^{2}_{o})$ which
represents the position of an advocate as the prior corresponds to the posterior
distribution based on the original data $\{\that_{o}, \sigma_{o}\}$ and a flat
prior for the effect size $\theta$. This is done by computing the Bayes factor
\begin{align*}
  \BF_{SA}
  = \frac{f(\that_{r} \given H_{S})}{f(\that_{r} \given H_{A})}
  = \sqrt{\frac{\sigma^{2}_{o} + \sigma^{2}_{r}}{\sigma^{2}_{s} + \sigma^{2}_{r}}}
  \, \exp\left[-\frac{1}{2}\left\{\frac{\that_{r}^{2}}{\sigma^{2}_{s} +
  \sigma^{2}_{r}} - \frac{(\that_{r} - \that_{o}^{2})}{\sigma^{2}_{o} +
  \sigma^{2}_{r}}\right\}\right]
\end{align*}
and replication success at level $\gamma$ is defined by $\BF_{SA} \leq \gamma$
as the data favor the advocate over the sceptic at a higher level than the
sceptic's initial objection to the null hypothesis. The smallest level $\gamma$
at which replication success is achievable is then called \emph{the sceptical
  Bayes factor} $\BFs$, and replication success at level $\gamma$ is equivalent
to $\BFs \leq \gamma$ \citep[see][for details on how to compute
$\BFs$]{Pawel2022b}. To derive the success region of the sceptical Bayes factor
one can apply algebraic manipulations to $\BF_{SA} \leq \gamma$ the condition
for replication success at level $\gamma$, leading to
\begin{align}
  \label{eq:BFssuccess}
  S_{\scriptscriptstyle \BFs}
  = \begin{cases}
    (-\infty, -\sqrt{B} - M] \bigcup [\sqrt{B} - M, \infty) & \text{for} ~ \sigma^{2}_{s} < \sigma^{2}_{o} \\
    [\that_{o} -\{(\sigma^{2}_{o} + \sigma^{2}_{r})\log\gamma\}/\that_{o}, \infty)
    & \text{for} ~ \sigma^{2}_{s} = \sigma^{2}_{o}  \\
    [-\sqrt{B} - M, \sqrt{B} - M] & \text{for} ~ \sigma^{2}_{s} > \sigma^{2}_{o}
    \end{cases}
\end{align}
with
\begin{align*}
  B &= \left\{\frac{\that_{o}^{2}}{\sigma^{2}_{o} - \sigma^{2}_{s}} +
      2\log\left(\frac{\sigma^{2}_{o} + \sigma^{2}_{r}}{\sigma^{2}_{s} + \sigma^{2}_{r}}\right)
      - 2\log \gamma \right\}
      \frac{(\sigma^{2}_{s} + \sigma^{2}_{r})(\sigma^{2}_{o} + \sigma^{2}_{r})}{\sigma^{2}_{o}
       - \sigma^{2}_{s}} \\
  M &= \frac{\that_{o} (\sigma^{2}_{s} + \sigma^{2}_{r})}{\sigma^{2}_{o} - \sigma^{2}_{s}}
\end{align*}
and the sufficiently sceptical prior variance $\sigma^{2}_{s}$ computed
by~\eqref{eq:ssv}.


%% Bibliography
%% -----------------------------------------------------------------------------
\bibliographystyle{../apalikedoiurl}
\bibliography{../bibliography}

\end{document}
